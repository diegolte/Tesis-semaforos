%La línea de abajo es para quitar encabezado
%\thispagestyle{plain}

\chapter*{Introducción}
\markboth{Introducción}{Introducción}
\addcontentsline{toc}{chapter}{Introducción}

En las ciudades modernas, la gestión del tráfico vehicular ha evolucionado de un desafío simple a un problema complejo que impacta directamente en la calidad de vida de los ciudadanos. La congestión del tráfico afecta a la movilidad, la productividad económica, la eficiencia del transporte público, la salud mental y la calidad ambiental debido a las emisiones de gases contaminantes. En muchas ciudades, los semáforos aún funcionan con sistemas estáticos y preestablecidos, que no responden dinámicamente a las condiciones reales del tráfico, resultando en tiempos de espera innecesarios y un flujo vehicular ineficiente. Esta ineficiencia se vuelve aún más crítica durante las horas pico, donde la acumulación de vehículos en intersecciones genera largos atascos y retrasa el transporte.

La aplicación de tecnologías de inteligencia artificial (IA) se presenta como una solución viable para optimizar la gestión del tráfico, aprovechando técnicas avanzadas como el aprendizaje por refuerzo (RL). Este enfoque tiene la capacidad de adaptarse en tiempo real a las condiciones cambiantes del tráfico, aprendiendo de las interacciones con el entorno para tomar decisiones que maximicen el flujo vehicular y minimicen los tiempos de espera en las intersecciones. Al aplicar esta tecnología a los semáforos, se puede crear un sistema inteligente que ajuste la sincronización de los semáforos de manera dinámica, mejorando la eficiencia del tráfico en zonas urbanas con alta congestión.

El distrito de Santiago de Surco, en Lima, Perú, es un ejemplo representativo de una zona urbana con una infraestructura de semáforos que aún depende de la sincronización manual y estática, lo que genera largas demoras en las horas pico. Este trabajo tiene como objetivo diseñar e implementar un sistema de gestión inteligente de semáforos utilizando aprendizaje por refuerzo, con el fin de optimizar la sincronización de los semáforos y mejorar el flujo vehicular en este distrito. Se propone que la implementación de este sistema no solo reducirá los tiempos de espera, sino que también contribuirá a la disminución de la congestión vehicular, mejorando la movilidad urbana y, por ende, la calidad de vida de los habitantes.

A lo largo de esta investigación, se explorarán los beneficios del aprendizaje por refuerzo como una herramienta para la toma de decisiones en tiempo real y su aplicabilidad a la gestión del tráfico, proporcionando un enfoque innovador para resolver problemas urbanos de movilidad. Asimismo, se evaluará la efectividad de esta tecnología mediante la simulación de escenarios reales, comparando su desempeño con los sistemas de semáforos tradicionales. Finalmente, se presentarán las conclusiones sobre la viabilidad y los impactos potenciales de la implementación de semáforos inteligentes basados en IA en el contexto del tráfico urbano de Santiago de Surco.
