\subsection{Inteligencia artificial en sistemas de transporte urbano}
El uso de la inteligencia artificial (IA) en sistemas de transporte urbano está transformando la manera en que las ciudades abordan problemas de movilidad y sostenibilidad. Según \cite{Nikitas2020}, la IA permite optimizar el transporte urbano al procesar grandes cantidades de datos en tiempo real, lo que mejora la eficiencia operativa y reduce los impactos ambientales como las emisiones de CO2. Esta tecnología integra enfoques como redes neuronales artificiales, lógica difusa y algoritmos evolutivos, que ayudan a resolver problemas de congestión y a gestionar el tráfico de manera dinámica en sistemas urbanos complejos.

La implementación de IA en el transporte urbano incluye aplicaciones como el control de señales de tráfico inteligentes, vehículos autónomos y sistemas de movilidad como servicio (MaaS). Estas tecnologías utilizan datos de sensores, GPS y cámaras para analizar patrones de tráfico y predecir comportamientos futuros. Esto permite decisiones en tiempo real, como ajustar las fases de los semáforos en función del volumen vehicular o priorizar el transporte público en horarios específicos.

Uno de los avances más destacados en este campo es la integración de vehículos autónomos y conectados, que utilizan algoritmos de aprendizaje profundo para tomar decisiones de navegación y evitar colisiones. Según \cite{Abduljabbar2019}, estas soluciones han demostrado ser efectivas para reducir tiempos de viaje en un 25\% y disminuir las emisiones de gases contaminantes hasta en un 30\%. Además, la IA facilita la coordinación entre diferentes modos de transporte, promoviendo una movilidad más eficiente y sostenible en las ciudades.

A pesar de sus beneficios, la adopción de IA enfrenta desafíos como la privacidad de los datos, la interoperabilidad tecnológica y la resistencia al cambio por parte de los usuarios y operadores. Sin embargo, el impacto positivo en la calidad de vida, la sostenibilidad y la productividad económica convierte a la IA en una herramienta esencial para las ciudades del futuro, transformando la gestión del tráfico en un componente clave de las ciudades inteligentes.

\subsection{Aprendizaje por refuerzo y su aplicación en la gestión de tráfico}
El aprendizaje por refuerzo (RL, por sus siglas en inglés) se ha establecido como una herramienta poderosa para resolver problemas complejos de gestión de tráfico en entornos urbanos. Según \cite{Greguric2020}, el RL permite abordar la congestión persistente en redes de tráfico densas mediante la optimización de los sistemas de control de señales de tráfico. A diferencia de los enfoques tradicionales que dependen de modelos empíricos predefinidos, el RL utiliza un enfoque basado en prueba y error para ajustar las decisiones de control en tiempo real, lo que resulta en una gestión más adaptable y eficiente.

Una de las aplicaciones más destacadas del aprendizaje por refuerzo en el tráfico es el control adaptativo de señales de tráfico (ATSC). Este enfoque utiliza datos en tiempo real provenientes de sensores y cámaras para modelar los estados del tráfico y predecir patrones futuros. El RL permite a los agentes (como los semáforos) ajustar dinámicamente los ciclos de luz verde y roja, maximizando el flujo vehicular y minimizando los tiempos de espera. Modelos avanzados, como el aprendizaje profundo por refuerzo (DRL), han demostrado ser particularmente efectivos en la resolución de problemas de alta dimensionalidad al integrar redes neuronales profundas que aproximan funciones de calidad y mejoran las decisiones del agente según \cite{Sarwar2023}.

En un estudio reciente, Sarwar et al. implementaron un modelo de aprendizaje por refuerzo multi-agente (MARL) para optimizar las señales de tráfico en redes urbanas. Este enfoque descentralizado permite que múltiples intersecciones colaboren, compartiendo información sobre el tráfico local y ajustando las decisiones de control de manera coordinada. Los resultados mostraron una reducción del 20\% en el tiempo promedio de espera por vehículo y un aumento en la capacidad de procesamiento del tráfico, destacando la robustez de MARL frente a los enfoques tradicionales \cite{Zourbakhsh2022}.

\subsection{Simulación computacional en estudios de tráfico vehicular}
La simulación computacional ha surgido como una herramienta esencial para abordar problemas complejos en la gestión del tráfico vehicular, permitiendo evaluar y optimizar la infraestructura vial y los patrones de movilidad en entornos urbanos y rurales. Según \cite{Alghamdi2022}, los modelos de simulación pueden replicar el comportamiento del tráfico en diferentes escalas (microscópica, macroscópica y mesoscópica), lo que los convierte en herramientas versátiles para analizar desde intersecciones específicas hasta redes viales completas. Estas simulaciones no solo mejoran la precisión en la predicción de congestionamientos, sino que también son fundamentales para planificar redes de transporte más eficientes y sostenibles.

En el ámbito microscópico, los simuladores como SUMO y VISSIM modelan el comportamiento individual de vehículos, permitiendo analizar interacciones específicas como el seguimiento vehicular y las decisiones de cambio de carril. Estas herramientas, según \cite{Hao2024}, incorporan algoritmos avanzados de aprendizaje profundo, como redes convolucionales y unidades de memoria recurrentes, para predecir con alta precisión patrones de tráfico. En experimentos recientes, un marco basado en datos alcanzó una precisión del 97.22\% durante el entrenamiento y del 95.76\% durante las pruebas, demostrando su capacidad para gestionar escenarios complejos.

En un enfoque macroscópico, los modelos de simulación fluidodinámica, como CORSIM y PARAMICS, evalúan el flujo de tráfico a gran escala, considerando variables como la densidad vehicular y la velocidad promedio. Estos modelos son ideales para optimizar la distribución del tráfico en corredores urbanos y carreteras. Por otro lado, los modelos mesoscópicos, que combinan características microscópicas y macroscópicas, son útiles para representar el tránsito en redes de mediana complejidad.


\subsection{Indicadores clave para evaluar la eficiencia del tráfico vehicular}  
Los indicadores clave de desempeño (KPIs) son herramientas esenciales para medir la eficiencia de los sistemas de tráfico vehicular, proporcionando información cuantitativa sobre el rendimiento de las estrategias de gestión del tráfico. Según \cite{Ahsini2023}, los KPIs permiten analizar aspectos como la velocidad promedio, los tiempos de espera y la densidad vehicular, ofreciendo una base para implementar mejoras en redes urbanas y evaluar el impacto de políticas de movilidad sostenible. Estos indicadores son fundamentales en ciudades inteligentes, donde la toma de decisiones informada es clave para garantizar un tráfico eficiente y sostenible.

Uno de los principales indicadores utilizados es el \textit{tiempo promedio de espera por vehículo}, que mide la eficiencia en intersecciones y corredores críticos. En un estudio realizado en Barcelona, este KPI permitió identificar cuellos de botella y optimizar rutas alternativas, logrando una reducción del 15\% en los tiempos de espera promedio al aplicar soluciones basadas en datos abiertos. Además, otros indicadores como la \textit{longitud de las colas} y la \textit{proporción de uso del transporte público} son utilizados para evaluar la congestión y promover alternativas sostenibles, como el transporte colectivo y la movilidad activa.

En términos de sostenibilidad, los KPIs relacionados con las \textit{emisiones de CO\textsubscript{2}} y el consumo de combustible se utilizan para medir el impacto ambiental de los sistemas de tráfico. Estos indicadores son especialmente relevantes en proyectos que buscan minimizar la huella de carbono del transporte urbano. Según el estudio, una integración adecuada de datos puede reducir las emisiones en un 20\% al priorizar modos de transporte más ecológicos y fomentar el uso de vehículos eléctricos en áreas de alta densidad vehicular.

\subsection{Sistemas de semáforos adaptativos basados en tecnología inteligente}  
Los sistemas de semáforos adaptativos representan una solución innovadora para mitigar la congestión vehicular en entornos urbanos. Según \cite{Zourbakhsh2022}, estos sistemas utilizan tecnologías como el Internet de las Cosas (IoT) y el aprendizaje por refuerzo para ajustar dinámicamente las fases de los semáforos en función de las condiciones de tráfico en tiempo real. Este enfoque permite mejorar significativamente el flujo vehicular y reducir los tiempos de espera en las intersecciones, especialmente en redes urbanas densas, donde los sistemas tradicionales de control fijo resultan ineficientes.

Un aspecto clave de estos sistemas es la integración de sensores y cámaras de vigilancia, que recopilan datos en tiempo real sobre el volumen de vehículos y patrones de tráfico. Estos datos se procesan mediante algoritmos avanzados, como el aprendizaje por refuerzo multiagente (MARL), que permiten a las intersecciones colaborar y coordinar sus decisiones. En el caso de la ciudad de Shiraz, Irán, este enfoque logró reducir las longitudes de las colas en las intersecciones en un 25\% y los tiempos de espera promedio en un 30\%, en comparación con los sistemas de semaforización fija, según \cite{Zourbakhsh2022}.

Además, el uso de técnicas como el aprendizaje profundo refuerza la capacidad de los semáforos para predecir condiciones futuras del tráfico y ajustarse a variaciones repentinas en la demanda vehicular. Según \cite{Zhang2024}, la implementación de algoritmos avanzados, como el Deep Q-Network (DQN) y sus variantes, ha demostrado ser eficaz para optimizar el control de semáforos en entornos simulados y reales, logrando una mejora del 40\% en el flujo vehicular en escenarios altamente congestionados.